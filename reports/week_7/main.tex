\documentclass{article}
\usepackage[utf8]{inputenc}

\title{Week 7 report}
\author{Javier Torres}
\date{February 13-17, 2017}

\begin{document}

\maketitle

\tableofcontents

\section{Advance Status}

\paragraph{Printed Circuit Board (PCB)}
I have finished mounting and testing the board. All the changes that needed to be made were applied to the final design.
For testing purposes I've made all the essential changes on the prototype so it will work correctly: Connect the MCU to 5v instead of 12v and detach the SLP pin of the motor controller from the MCU. With this changes applied, the system works as intended.

\section{Comments and Thoughts}

\paragraph{About the mounting and testing process}
This process was much slower than expected, mainly because of the design itself. When I was designing the board I didn't had in mind an easy mounting process. Much less to make it easier for me to perform modifications on the board during the mounting process. I will keep this in mind for future designs. This process could have taken much less time than it did.

\paragraph{About the burned out components}
As it was my fault, I have purchased and replaced the components that were damaged during last week.
\section{Next Steps}

\paragraph{Final PCB manufacture}
A second (and final) version of the PCB was designed. It will be manufactured as soon as possible.

\paragraph{Prototype enclosure}
I will design and print an enclosure for all the components.

\paragraph{Project documentation}
I will prepare a document with all the relevant information about the project: Operational manual, data-sheet and others.

\paragraph{Project assembly}
As soon as the PCB's are manufactured and the enclosure is printed, I will solder all the components on the PCB and mount everything inside the case.

\end{document}