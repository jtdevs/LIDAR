\documentclass{article}
\usepackage[utf8]{inputenc}

\title{Week 5 report}
\author{Javier Torres}
\date{January 30 - February 3, 2017}

\begin{document}

\maketitle

\tableofcontents

\section{Advance Status}

\paragraph{Software update}
I have modified the program to fit the connections made during the PCB design process. Also, new features were added, such as master/slave selection, debug mode activation (using hardware input) and dynamic degrees-per-step selection for the stepper motor.

\paragraph{PCB Manufacture}
Regarding the PCB Manufacture, the following steps will be taken:
\begin{itemize}
\item Manufacture a prototype board, without silkscreen or anti-solder mask.
\item Solder all the components to the board and test.
\item Evaluate if any modifications are necessary.
\item Manufacture two or more final versions of the board with all the possible modifications included.
\end{itemize}
Where each board will be manufactured is still to be decided.

\section{Comments and Thoughts}

\paragraph{About the PCB manufacturing options}
Two options are being evaluated to manufacture the boards: AC3E's recommended provider or Tectronix.
Tectronix is a local electronic component store that has a PCB manufacturing service intended for rapid prototyping of a low number of units. 
AC3E's recommended provider (whose name I still don't know) is a company located in China that has a PCB manufacturing service. As the company is located abroad, the boards take longer to be in our hands and its service is more suited for a high volume order (more than 5 units).
My proposal is to manufacture the prototype board at Tectronix and the final version at AC3E's recommended provider.

\section{Next Steps}

\paragraph{PCB manufacture and testing}
Once the PCB prototype is in my hands I will solder all the electronic components on the PCB and look for possible improvements, mainly in the distribution of the components and test that everything is working as intended.

\paragraph{Prototype enclosure}
Once all the electronic components are mounted on the PCB and everything is working properly, I will start working on an enclosure for the project.

\end{document}
